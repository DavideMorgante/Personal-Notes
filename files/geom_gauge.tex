\chapter{Geometry of gauge theories}
In this chapter we summarise the relevant mathematical basics for the gometrization of gauge theories. 
\section{Differential geometry basics}
To define gauge fields geometrically, we need the notion of forms. Here we give a very basic defintion of them and the usual operations they carry. 

\begin{defn}{Differential Form}{}
    Given a differentiable $d$-manifold $X$, a \textit{differential form} on $X$ is a section\footnote{Fancy for function} of the exterior algebra of the cotangent bundle over $X$. A differential $p$-form, with $p<d$, on $X$ is a section of the $p$th exterior power of the cotungent bundle. The integer $p\in\bbZ_{>}$ is called \textit{rank} of the form.
\end{defn}

The set of differential $p$-forms is denoted by $\Omega^p(X)$. The basic differential forms are the line element forms $\dd{x}$ for some set of coordinates $\{x\}$. On this basis, any form can be decomposed 
\begin{equation}
    \omega_p=\sum_if_{i_1,\ldots,i_p}(x)\dd{x_{i_1}}\wedge\dd{x_{i_p}}
\end{equation}
Forms are the "dual" of vector fields, meaning that there is a natural inner product such that 
\begin{equation}
    \langle \vec{v},\omega\rangle=\sum_i v^i(x)f_i(x),\qquad \vec{v}=\sum_i v^i \pdv{}{x^i}
\end{equation}
The wedge product of two forms with degrees $p,q$ such that $p+q\le d$ is graded commutative, meaning 
\begin{equation}
    \alpha_p\wedge \beta_q=(-1)^{pq}\beta_q \wedge \alpha_p
\end{equation}
By construction, differential forms are integrable 
\begin{equation}
    \oint_{\Sigma_n}\omega_n=\sum_{i_1,\ldots,i_n}\oint_{\Sigma_n}f_{i_1,\ldots,i_n}\dd{x}_1\ldots\dd{x}_n
\end{equation}
where the pullback of the volume form to $\Sigma_n$ vanishes except when it is proportional to the volume element on $\Sigma_n$.

On forms we have exterior derivative which increases the degree of the form and squares to zero. Exterior derivatives obey the product rule 
\begin{equation}
    \dd{(\alpha_p\wedge \beta_q)}=\dd{\alpha}_p\wedge \beta_q+(-1)^p\alpha_p\wedge \dd{\beta}_q
\end{equation}
\begin{thm}{Generalized Stokes' theorem}{}
    Let $\Sigma$ be an oriented $(n+1)$-manifold with boundary $\partial \Sigma$ and let $\omega_n$ being a differential $n$ form. Then the integral of $\dd{\omega}_n$ is given by the integral of $\omega_n$ on $\partial \Sigma$ 
    \begin{equation}
        \int_\Sigma \dd{\omega}_n=\int_{\partial \Sigma}\omega_n
    \end{equation}
\end{thm}
With the exterior differential we can construct the de Rham cohomology $\HH_\rm{dR}^n(M)=\HH^n(M;\bbR)$. Interesting in itself is the restriction of the cohomology to forms with integral periods $\Omega_\bbZ^n(M)$, i.e. they integrate to an integer on all $n$-dimensional submanifolds of $M$. From this one can construct the integral-valued cohomology classess $\HH^n(M;\bbZ)$ which have a similar definition to the continuous case. 

\section{Gauge theories}
A gauge theory is the theory of principal $G$ bundles with connection where $G$ is some Lie group. Principal $G$ bundles are fibrations of $G$ over some base manifold 
\begin{equation}
\begin{tikzcd}
    G \arrow{r} & P \arrow{d}{\pi} \\
    & X 
\end{tikzcd}
\end{equation}
where $\pi$ is the projection onto the base manifold. Let us describe how to construct $G$-bundles with $P$ connection. Let us fix $G$ to be a semi-simple Lie group. A $G$-bundle $P$ over $X$ can be locally trivialized by taking a nice covering of open sets over $X$. Over each $U_\alpha$, the fiber of $P$ looks like $\pi^{-1}(U_\alpha)=P_\alpha\simeq G\times U_\alpha$. The fibers of $P$ are related between patches by transition functions (gauge transformations)
\begin{equation}
    g_{\alpha \beta}:U_\alpha\cap U_\beta\rightarrow G,\qquad (x,p_\alpha)=(x,g_{\alpha \beta}(x)p_\beta),\quad x\in U_\alpha\cap U_\beta
\end{equation}
All fiber bundles obey the triple intersection condition 
\begin{equation}
    g_{\alpha \beta }(x)g_{\beta \gamma }(x)g_{\gamma \sigma}(x)=\delta_{\alpha \sigma}\qquad \forall x\in U_\alpha\cap U_\beta \cap U_\sigma
\end{equation}
This condition can be modified for twisted fiber bundles.

A connection on $P$ is a rule for comparing fibers at different points $\pi^{-1}(x),\phi^{-1}(y)$ for $x\neq y$. The point is that parallel transport around the base manifold can be identified with a tangent vector field to $X$ and, for non-trivial $P$, motion along such vector field can induce a transition in the fiber. This translation is itself described by a vector that is tangent to the fiber $G$ that is a function of the tangent vector field $X$. Thus the translation in $G$ is locally described by a $\mathfrak{g}$-valued one-form which is called the connection $A$. The triple intersection condition implies that the restriction of the connection $A$ to an open cover are related by the gauge transformation 
\begin{equation}
    A_\beta=g^{-1}_{\alpha \beta} A_\alpha g_{\alpha \beta}+ g_{\alpha \beta}^{-1}\dd{g}_{\alpha \beta}
\end{equation}
From this one can construct the covariant derivative 
\begin{equation}
    D_A=\delta+A
\end{equation}
This connection can have non-trivial curvature, the field strength, which is given by the commutator of the covariant derivatives 
\begin{equation}
    F=\dd{A}+A\wedge A=D_A A
\end{equation}

\subsection{Some Details}
The key is Weyl's famous observation that electrodynamics is really (classical) $U(1)$-gauge theory. Concretely: 
\begin{itemize}
    \item You generalise the global $1$-form $A$ on $M$ to a connection $D_A$ on a Hermitian line bundle $\mathcal{L} \to M$, which can locally be written as $d + A$ for $A$ the so-called connection $1$-form.
    \item The differential $F := \mathrm{d}A$ of the global $1$-form $A$ is replaced by the curvature $$F := \mathrm{d}A + A \wedge A = D_A A$$ of the connection $D_A$, which is still a global $2$-form and still satisfies $\dd F = 0$ by the Bianchi identity as applied to a connection on a line bundle. 
    \item Gauge symmetry in this context now still holds, since the curvature $2$-form $F$ is unchanged if you replace $D_A$ by $D_A + \dd f$ for $f \in C^\infty(M)$.
\end{itemize}
But in general it can happen that $H^2(M,\bbR)\neq 0$ and therefore $F\neq d A$. The assignment
\begin{equation}
 (\text{line bundle $\mathcal{L} \to M$}) \mapsto (\text{curvature $2$-form $\mathcal{F}$ of a connection $\nabla$ on $\mathcal{L}$})
\end{equation}
induces a homomorphism
\begin{equation}
 \operatorname{Pic}(M) \to H^2_{\mathrm{dR}}(M) \cong H^2(M,\mathbb{R}),
\end{equation}
where the Picard group $\operatorname{Pic}(M)$ is the abelian group of isomorphism classes of line bundles on $M$, with operations defined as
\begin{equation}
 [\mathcal{L}] + [\mathcal{L}^\prime] := [\mathcal{L} \otimes \mathcal{L}^\prime], \quad -[\mathcal{L}] := [\mathcal{L}^\ast];
\end{equation}
Then $H^2(M,\mathbb{R}) = 0$ if and only if every closed $2$-form on $M$ is exact (i.e., $F = \dd A$ for some global $1$-form $A$), if and only if every line bundle is trivial or torsion (so that, necessarily, $D_A = d + A$ for a global $1$-form $A$). The moment that $H^2(M,\mathbb{R}) \neq 0$, however, you do have non-trivial line bundles and non-exact closed $2$-forms, so that you really do need to consider your spacetime $M$ together with a potentially non-trivial line bundle $\mathcal{L} \to M$.
Some comments are in order
\begin{itemize} 
    \item The cohomology group $H^2(M,\mathbb{R})$ contains an isomorphic copy of $H^2(M,\mathbb{Z})/\operatorname{Tor}(H^2(M,\mathbb{Z}))$ (via the UCT) as a lattice. It’s a non-trivial consequence of the Chern–Weil theory that our homomorphism $\operatorname{Pic}(M) \to H^2(M,\mathbb{R})$ not only maps into this lattice but actually recovers the Chern class $\operatorname{Pic}(M) \to H^2(M,\mathbb{Z})$ modulo torsion.
    \item Given a line bundle $\mathcal{L}$, a connection $\nabla$ on $\mathcal{L}$ is the *gauge potential* of an electromagnetic field in that topological sector, and the curvature $F$ of $\nabla$ is the *field strength* of that electromagnetic field.
    \item The class of the line bundle $\mathcal{L}$ in $\operatorname{Pic}(M) \cong H^2(M,\mathbb{Z})$ is called the *topological charge* or *topological defect*. If $H^2(M,\mathbb{Z}) \cong \mathbb{Z}$, then, in suitable units, the integer corresponding to $[\mathcal{L}]$ can be interpreted as a *monopole charge* à la Dirac. Indeed, the Dirac monopole can be interpreted as a certain connection on a certain non-trivial line bundle on $M = \mathbb{R}^{1,3} \setminus \text{(timelike worldline)}$.
\end{itemize}


\section{Characteristic classes}
\begin{proposition}
    For a fixed $G$, any principle $G$-bundle over a fixed base manifold $X$ can be constructed by pulling back a $G$ bundle from a universal spaced called the classifying space $BG$. The classifying space can be constructed as follows: pick a contractible space $EG$ with a free $G$-action. $BG$ can then be constructed as a quotient $EG/G$. Now any principal $G$-bundle $P\rightarrow X$ can be constructed by pulling back the bundle $EG\rightarrow BG$ 
    \begin{equation}
        \begin{tikzcd}
            G \arrow{r}{} & P  \arrow[swap]{d}{\pi} &\arrow[swap]{l}{f^*}EG \arrow{d}{\pi} &\arrow{l} G \\
            &X\arrow{r}{f}&BG & &
    \end{tikzcd}
    \end{equation}
\end{proposition}
So in particular, the cohomology of $BG$ classifyes the possible topology of principal $G$-bundles.

Characteristic classes classify the possible topological invariants associated with a gauge bundle, and as such must be constructed from gauge invariant differential written in terms of the field strength and the gauge field. When $G$ is a simply connected Lie group, the cohomology classes are given by guage-invariant polynomials in $F$ which are nothing but the Chern characters of $BG$ 
\begin{equation}
    \frac{\Tr F\wedge\cdots\wedge F}{n! (2\pi)^n}\in \HH^{2n}(BG;\bbZ)
\end{equation}
Chern classes are just reduction of Chern characters by products of lower-dimensional Chern-classes, for example 
\begin{equation}
    c_1(F)=\rm{ch}_1(F)=\frac{\Tr F}{2\pi},\qquad c_2(F)=\rm{ch}_2(F)-\frac{1}{2}c_1(F)^2,\qquad\ldots
\end{equation}
Some examples of these cohomology groups are the following 
\begin{equation}
    \HH^{2n}(B\U(1);\bbZ)=\bbZ\qty[\frac{\Tr F\wedge\cdots\wedge F}{n! (2\pi)^n}],\qquad H^{2n}(B\bbZ_N;\bbZ)=\bbZ_N
\end{equation}
For non-simply connected Lie groups like $\SO(N)$ or $\PSU(N)$, we have a relation due to the Universal Coefficient Theorem
\begin{equation}
    \HH_1(G;\bbZ)\simeq \HH^2(BG;\bbZ)
\end{equation}
This means that a $G$-gauge theory has discrete $2$-form fluxes that correspond to the first homotopy group of $G$ which encodes the $1$-form magnetic flux sectors.
\section{Twisted bundles and Center symmetry}




\section{Line bundles}
A section is just a function which takes values on different vector spaces at each point of the target space. This means that in general $\phi:M\rightarrow \bbC^N$ the output of the function might not be the same $\bbC^N$ at each point of $M$. 

The simplest example of line bundle is the trivial line bundle $\bbC\times M$. Here the vector space at each point is just $\{m\}\times\bbC$ where $m$ is the point in $M$.

\begin{defn}{Complex line bundle}{}
    A complex line bundle over a manifold $M$ is a manifold $L$ and a smooth surjection $\pi:L\rightarrow M$ such that
    \begin{itemize}
        \item Each fiber $\pi^{-1}(m)=L_m$ is a complex one-dimensional vector space $L_m\simeq \bbC$
        \item Every $m\in M$ has an open neighborhood $U\subset M$ for which there is a diffeomorphism 
        \begin{equation}
            \phi:\pi^{-1}(U)\rightarrow U\times \bbC\qquad\text{such that }\phi(L_m)\subset\{m\}\times\bbC
        \end{equation}
        for every $m$ and that moreover the map 
        \begin{equation}
            \phi\big|_{L_m}:L_m\rightarrow \{m\}\times \bbC
        \end{equation}
        is a linear isomorphism. The second condition is known as local triviality.
    \end{itemize}
\end{defn}
In quantum mechanics, local triviality means that at least in some local region like the laboratory we can identify the Hermitian vector space where the wave function takes its values in $\bbC$.

A section of a line bundle $L$ is like a vector field. That is, a map $\phi:M\rightarrow L$ such that $\phi(m)\in L_m$ $\forall m$, or more succintly $\pi\circ\phi=\rm{id}_m$. The set of all sections, denoted by $\Gamma(L,M)$, is a vector space under pointwise addition and scalar multiplication. A line bundle is trivial iff it has a nowhere vanishing section.

A locally trivial line bundle can be completely understood from the transition function on the base manifold. However, the transition functions are by no means unique. Even by fixing a covering on $M$ we could replace each coordinate $s_\alpha$ on $U_\alpha$ by $h_\alpha s_\alpha$ for some $h_\alpha:U_\alpha\rightarrow \bbC^\times$. Then $g_{\alpha \beta}$ becomes $h_\alpha g_{\alpha \beta }h_\beta^{-1}$. To understand this ambiguity one needs to look at Čech cohomology since, it can be shown, that isomorphism classes of complex line bundles over $M$ are in bijection with the Čech cohomology group $\check{\HH}^1(M,\bbC^\times)$.
\begin{defn}{Connection}{}
    A connection $\nabla$ over a line bundle is a linear map 
    \begin{equation}
        \nabla:\Gamma(M,L)\rightarrow \Gamma(M, T^* M\otimes L)
    \end{equation}
    such that $\forall s\in \Gamma(M,L)$ and $f\in C^\infty(M,L)$ we have the Liebnitz rule 
    \begin{equation}
        \nabla(fs)=\dd{f}\otimes s+f\nabla s
    \end{equation}
\end{defn}

Every line bundle has a connection. Moreover, if we have a connection over a line bundle there always exists a unique connection over the restriction on open covers. \\
Let $L\rightarrow M$ be a line bundle ans $s_\alpha:U_\alpha\rightarrow L$ being a local nowhere vanishing section. Define a one-form $A_\alpha$ on $U_\alpha$ by 
\begin{equation}
    \nabla s_\alpha=A_\alpha\otimes s_\alpha
\end{equation}
If $\xi\in \Gamma(M,L)$ then $\xi|_{U_\alpha}=\xi_\alpha s_\alpha$ where $\xi_\alpha:U_\alpha\rightarrow \bbC$ and 
\begin{equation}
    \nabla \xi\big|_{U_\alpha}=\dd{\xi_\alpha}s_\alpha+\xi_\alpha \nabla s_\alpha=(\dd{\xi_\alpha}+A_\alpha \xi_\alpha)s_\alpha
\end{equation}
Recall that the transition functions on the base manifold lift to the ones of the line bundle and $s_\beta=g_{\alpha \beta}s_\alpha$, so $\nabla s_\alpha=\dd{g}_{\alpha \beta}s_\beta+g_{\alpha \beta}\nabla s_\beta$ and hence $A_\alpha s_\alpha=g^{-1}_{\alpha \beta} \dd{g}_{\alpha \beta}g_{\alpha \beta}s_\alpha+s_\alpha A_\beta$. Therefore 
\begin{equation}
    A_\alpha=A_\beta+g^{-1}_{\alpha \beta}\dd{g}_{\alpha \beta}
\end{equation}
The converse is also true. If $\{A_\alpha\}$ is a collection of one-forms satisfying the last equation on $U_\alpha\cap U_\beta$ then there is a connection $\nabla$ such that $\nabla s_\alpha=A_\alpha s_\alpha$.

At this point, if we have a curve $\gamma:\comm{0}{1}\rightarrow M$ and a connection $\nabla$, to parallel transport a section $\xi(t)\in L_{\gamma(t)}$ we require that 
\begin{equation}
    \nabla_{\dot{\gamma}}\xi=0
\end{equation}
or, equivalently taking $\xi(t)=\xi_\alpha(t)s_\alpha(\gamma(t))$ we have 
\begin{equation}
    \frac{\dd{\xi_\alpha}}{\dd{t}}=-A_\alpha(\gamma)\xi_\alpha\leftrightarrow \xi_\alpha(t)=\exp\qty(-\int_0^t A_\alpha(\gamma(t)))\xi_\alpha(0)
\end{equation}
Parallel transport is an isomorphism 
\begin{equation}
    P_\gamma:L_{\gamma(0)}\rightarrow L_{\gamma(1)}
\end{equation}
When the curve is a loop $P$ is the holonomy of the connection over $\gamma$ 
\begin{equation}
    P_\gamma(s)=\rm{hol}(\gamma,\nabla)s
\end{equation}
If we gave a loop $\gamma:\comm{0}{1}\rightarrow U_\alpha$ then
\begin{equation}
    \rm{hol}(\gamma, \nabla)=\exp\qty(-\oint_\gamma A_\alpha)
\end{equation}
If $\gamma$ is the boundary of a disk $D$ then 
\begin{equation}
    \rm{hol}(\gamma,\nabla)=\exp\qty(-\int_D \dd{A}_\alpha)
\end{equation}
Consider the transformation of the one-form. This two form $\dd{A}_\alpha$ is therefore 
\begin{equation}
    \dd{A}_\alpha=\dd{A}_\beta+\dd{} \qty(g_{\alpha \beta}^{-1}\dd{g}_{\alpha \beta})=\dd{A}_\beta-g_{\alpha \beta}^{-1}\dd{g}_{\alpha \beta}g_{\alpha \beta}^{-1}\wedge \dd{g}_{\alpha \beta}+g_{\alpha \beta }\dd{\dd{g}}_{\alpha\beta}=\dd{A}_\beta
\end{equation}
so the two connections agree on the intersection and hence define a global two form we denote $F$ and call the curvature of $\nabla$. Then 
\begin{proposition}
    If $L\rightarrow M$ is a line bundle with connection $\nabla$ and $\Sigma$ is a compact submanifold of $M$ with boundary a loop $\gamma$ then 
    \begin{equation}
        \rm{hol}(\gamma, \nabla)=\exp\qty(-\int_\Sigma F)
    \end{equation}
\end{proposition}

Now we can define Chern classes which are topological invariant of a line bundle. 
\begin{proposition}
    The curvature $F$ of a connection $\nabla$ satisfies the following conditions 
    \begin{itemize}
        \item $\dd{F}=0$
        \item If $\nabla,\nabla^\prime$ are two connections, then $\nabla=\nabla^\prime +\eta$ for $\eta$ a one-form and $F_\nabla = F_{\nabla^\prime}+\dd{\eta}$
        \item If $\Sigma$ is a closed surface then 
        \begin{equation}
            \frac{1}{2\pi i} \int_\Sigma F_\nabla\in \bbZ
        \end{equation}
        independent of $\nabla$.
    \end{itemize}
\end{proposition}
\begin{defn}{Chern class}{}
    The Chern class $c(L)$ of a line bundle $L\rightarrow \Sigma$, where $\Sigma$ is a surface, is defined to be the integer 
    \begin{equation}
        \frac{1}{2\pi i}\int_\Sigma F_\nabla,\qquad \forall \nabla
    \end{equation}
\end{defn}

\begin{example}
    Consider the tangent bundle to $S^2$ $TS^2$. The curvature form on this line bundle is just $F=-i\rm{vol}_{S^2}$. It is easy to see than that 
    \begin{equation}
        c(TS^2)=\frac{-i}{2\pi i }\int_{S^2}\rm{vol}_{S^2}=\frac{-i}{2\pi i}4\pi=-2
    \end{equation}
    Some further insights on $S^2$ can be seen as follows. Consider a covering of $S^2$ by two open subsets $U_0,U_1$, the usual coverings by half spheres. Let $L\rightarrow S^2$ be given by the transition function on the annulus $g_{01}:U_0\cap U_1\rightarrow \bbC^\times$. Then a connection is a pair of two one-forms $A_0,A_1$ on $U_0,U_1$ respectively such that 
    \begin{equation}
        A_1=A_0+\dd{g}_{10}g_{10}^{-1}\qquad\text{on }U_0\cap U_1
    \end{equation}
    Take $A_0=0$ and $A_1$ any extension of $\dd{g}_{10}g_{10}^{-1}$. Then 
    \begin{equation}
        F=\begin{cases}
            \dd{A}_0=0&\text{on }U_0\\
            \dd{A}_1&\text{on }U_1
        \end{cases}
    \end{equation}
    To find $c(L)$ we use Stokes theorem 
    \begin{equation}
        \int_{S^2}F=\int_{U_1}F=\int_{\partial U_1}\dd{A}_1=\int_{\partial U_1}\dd{g}_{10}g_{10}^{-1}
    \end{equation}
    But the last part is just the winding number of $g_{10}$ which is $2\pi i$. This is because when we try to glue vector fields on the annulus of the sphere, going around the annulus both rotare by $\pi$ and so $g_{01}$ rotates by $2\pi$.\\
    Hence the Chern class of $L$ is just the winding number of $g_{01}$. Isomorphism classes of line bundles on $S^2$ are in one to one correspondence with the integers via the Chern class.
\end{example}
\begin{thm}{Gauss-Bonnet}{}
    For a general Riemann surface with genus $g$ $\Sigma_g$ we have
    \begin{equation}
        c(T\Sigma_g)=2-2g
    \end{equation}
\end{thm}

So far we have only defined the Chern class for a surface. To define it for general manifolds we need to recall the definition of de Rham cohomology. Given a manifold $M$ we have the de Rham complex 
\begin{equation}
    0\rightarrow \Omega^0(M)\rightarrow \Omega^1(M)\rightarrow \cdots \rightarrow \Omega^m(M)
\end{equation}
where $m=\dim M$ and $\Omega^p(M)$ is the space of $p$-forms on $M$. The horizontal maps are exterior derivatives $\dd$. Then $\dd^2=0$ and it makes sense to define the de Rham cohomology groups 
\begin{equation}
    \HH^p(M)=\frac{\ker\Omega^p(M)\rightarrow \Omega^{p+1}(M)}{\Im \Omega^{p-1}(M)\rightarrow \Omega^p(M)}
\end{equation}
Each $\HH^p$ is a finite-dimensional vector space if $M$ is compact or otherwise well behaved. For a general definition of the Chern class is just 
\begin{defn}{Chern class}{}
    Given a manifold $M$, the Chern class of the line bundle $L$ is just $c(L)\in \HH^2(M)$.
\end{defn}

If we want to generalize the notion of line bundle to the one of vector bundle, we just replace every $\bbC$ with $\bbC^n$ and every $\bbC^\times$ with $\rm{GL}(n,\bbC)$.