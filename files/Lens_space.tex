\chapter{Lens spaces and index}
Here we will discuss the definition and various properties of lens spaces and the localization of gauge theories on such manifolds. We will see that the lens space index is an interesting quantity since it can distinguish between different global structures of the same Lie algebra for a given gauge theory thanks to the presence of non-contractible cycle which support non-trivial principal bundles which are distinguishable by certain characteristic classes. These topological objects are the one which allow for different global structures for a given gauge theory.
\section{Lens space}
Let us start by giving some basic definition about the lens space $L(p,q)$ where $p,q$ are coprime integers. This space can be defined as a quotient of a $3$-sphere $S^3$, seen as an embedding in $\bbC^2$, by the following $\bbZ_p$ action
\begin{equation}
	(z_1,z_2)\rightarrow (\omega^q_p~z_1,\omega^{-1}_p~z_2),\qquad\text{where } \omega_p\equiv \ee^{\frac{2\pi\I}{p}}\in\bbZ_p
\end{equation}
{\color{red}{Consider adding definition as a Seifert fibration over a punctured Riemann surface}}\\
This action has no fixed points and therefore defines a smooth manifold. We're going to be mainly interested in the case of $q=1$ where now the on the manifold $L(r,1)$ the $\bbZ_r$ action just rotates the Hopf fibers of $S^3$. For $r>1$ this lens space is not simply connected since $\pi_1(L(r,1))=\bbZ_r$. Instead, the limiting case of $r=1$ is just $L(1,1)\simeq S^3$. 
