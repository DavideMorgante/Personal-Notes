\chapter{Seiberg--Witten theory}
\section{Susy algebra}
The susy algebra in $4d$ is generated by a certain number of Weyl spinors 
\begin{equation}
	Q_{\alpha}^{i},\quad\Tilde{Q}_{\dot\alpha, j},\qquad \alpha,\dot\alpha=1,2;\quad i,j=1,\ldots,\cN
\end{equation}
whose algebra is 
\begin{equation}
	\acomm{Q_{\alpha}^{i}}{\tilde{Q}_{\dot\beta,j}}=\delta^{i}_{j}\sigma^{\mu}_{\alpha\dot\beta}P_{\mu}
\end{equation}
where $\sigma^{\mu}=(1,\vec{\sigma})$ are Pauli matrices and $P_{\mu}$ is the four momentum. The rest of usual anticommutators are
\begin{equation}
	\acomm{Q_{\alpha}^{i}}{Q_{\beta}^{j}}=\acomm{\tilde{Q}_{\dot\alpha,i}}{\tilde{Q}_{\dot\beta,j}}=0
\end{equation}
moreover there are all the other bosonic symmetries like Poincarè and internal symmetries. This algebra can be extended when we have conformally invariant theories (in that case, the anticommutator of two different supercharges contains also the special conformal transformation generator). There could also be some central extension to the anticommutators of the same kind of supercharges.

Now we discuss irreducible representations of this algebra. We look at states of positive mass $m>0$ by going to the rest frame where $P_{\mu}=(M,0,0,0)$ and there fore the algebra is
\begin{equation}
	\acomm{Q_{\alpha}}{\tilde{Q}_{\dot\beta}}=\delta^{i}_{j}\delta_{\alpha\dot\beta}M,\qquad \acomm{Q_{\alpha}}{Q_{\beta}}=\acomm{\tilde{Q}_{\dot\alpha}}{\tilde{Q}_{\dot\beta}}=0
\end{equation}
So we have the algebra for $2\cN$ fermion creation and annhilation operators. An irreducible rep has dimension $2^{2\cN}$.

Consider the case of $m=0$. The best we can do is $P_{\mu}=(E,0,0,E)$, considering a particle moving in the $z$ direciton. Then the algebra looks like the following (using the usually normalized $\sigma^{z}$)
\begin{equation}
	\acomm{Q_{\alpha}^{i}}{\tilde{Q}_{\dot\beta,j}}=\delta^{i}_{j}\delta_{\alpha\dot\beta}\,\mqty(1&0\\0&0)_{\alpha\dot\beta},\qquad \acomm{Q_{\alpha}}{Q_{\beta}}=\acomm{\tilde{Q}_{\dot\alpha}}{\tilde{Q}_{\dot\beta}}=0
\end{equation}
The main change is that the unit matrix whose eigenvalues are both one has been changed by a matrix whose one of the eigenvalues is zero. In lorentz signature, $\tilde{Q}=Q^{\dagger}$ is the hermitian adjoint of $Q$. If we set $\alpha=\dot\beta=2$ we get
\begin{equation}
	\acomm{Q_{2}^{i}}{\tilde{Q}_{2,i}}=0\implies \acomm{Q_{2}^{i}}{Q^{\dagger}_{2,i}}=0
\end{equation}
with no implicit sum on $i$. We have an operator whose anticommutator with it's adjoint is zero. But the anticommutator of an operator with is adjoint is positive definite. Therefore the only possibility is that on these states with $M=0$ $Q_{2}=\tilde{Q}_{2}=0$.\\
So instead of $2\cN$ creation and annihilation operators, we jet just $\cN$. So an irreducible rep has dimension $2^{\cN}$. With CPT, sometimes, $2\times 2^{\cN}$. If $\cN$ is not small, notice that $2^{2\cN}>2\times 2^{\cN}$. When this is true, some funny things are going to happen.

Let us look at small values of $\cN$. Start with the massless $\cN=1$. A massless particle is in an irrep of the Poincarè group labelled by it's helicity $j$.
The creation and annihilation operator are spinors $j_{z}=\pm 1/2$ therefore they raise and lower the helicity by half a unit. CPT reverses the sign of helicity so if we have a multiplet with helicity $j_{1},j_{2}$, we also have the states with $-j_{1},-j_{2}$.\\
For $\cN=0,1$ we always need four states. Since we are not interested in theories of gravity, but only gauge theory, we will consider reps with helicity $|j|\le1$ and there are only two possible massless reps
\begin{equation}
\begin{split}
	&(-j_{2},-j_{1},j_{0},j_{1},j_{2})\\
	\text{Vector Multiplet: }&(-1,-1/2,0,1/2,1)=(1,1,0,1,1)\\
	\text{Chiral Multiplet: }&(-1,-1/2,0,1/2,1)=(0,1,2,1,0).
\end{split}
\end{equation}
This spectrum of chiaral multiplet has $4$ states which is how many we need for a massive multiplet. And in fact this multiplet can get a mass in a completely susy fashion. The vector multiplet cannot since a massive vector has helicity zero which is not in the multiplet, it can get a mass with Higgsing.


Consider now $\cN=2$ in the massless case. The basic state, considering CPT, is as follows
\begin{table}[H]
\begin{tabular}{c|cccc}
	\text{N.States}&1&2&1&\\
	\hline
	\text{Helicity}&-j&$-j+\frac{1}{2}$&-j+1&$-j+\frac{3}{2}$
\end{tabular}
\end{table}
CPT means, first of all, that the quantum states are real. Whatever is the algebra of observables, their rep in the Hilbert space is real from CPT. Half of the $Qs$ are zero. The non-zero ones are four $2Q,2\tilde{Q}$ hermitian, by combination of these. These four generate a Clifford algebra. {\color{red}{Qui la registrazione è tagliata, comincia a parlare dei multipletti corti e lunghi in base alle estensioni centrali della superalgebra.}}

\section{\texorpdfstring{Most basic $\cN=2$ renormalizable gauge theory}{Most basic N=2}}
Beside free chiral theory, the most basic theory is a theory of only vector multiplets. The field content is: a vector $A$, two fermions $\lambda^{1,2}$, and a complex scalar $\phi$. We can construct three theories with this: the first is by hand starting with $\cN=1$ (a $\cN=2$ hypermultiplet is a vector+chiral multiplet in $\cN=1$).

Write down simplest $\cN=1$ action and claim that actually is $\cN=2$
\begin{equation}
	\int \dd[4]{x}\dd[2]{\theta} \Tr W_{\alpha}W^{\alpha}=\frac{1}{e^{2}}\int\dd[4]x\Tr\qty(F_{\mu\nu}^{2}+i\bar\lambda^{1}\slashed{D}\lambda^{1}+D^{2})
\end{equation}
plus we have a part for the adjoint chiral $\Phi$
\begin{equation}
\begin{split}
	\int\dd[4]{x}\dd[4]\theta\Tr\bar\Phi e^{V}\Phi=\frac{1}{e^{2}}\int\dd[4]{x}\Tr\Big(&D_{\mu}\bar\phi D^{\mu}\phi+\bar\lambda^{2}\slashed{D}\lambda^{2}+D\comm{\phi}{\bar{\phi}}\\
	&+\comm{\bar\lambda^{1}}{\bar\lambda^{2}}\phi+\comm{\lambda^{1}}{\lambda^{2}}\bar\phi\Big)
\end{split}
\end{equation}
Remember that $\lambda$ has one chirality and $\bar\lambda$ has the opposite chirality. Just Lorentz invariance does not tell us if we have to take the Yukawa coupling of $\lambda s$ with $\phi$ or $\bar\phi$. This is fixed by internal symmetries $\U(1)\times\U(1)_{R}$
\begin{equation}
	\U(1):\Phi\rightarrow e^{i\alpha}\Phi
\end{equation}
fixes the right product. The $R$-symmetry acts on the superspace as
\begin{equation}
	x\rightarrow x,\quad \theta\rightarrow e^{i\beta}\theta,\quad\bar\theta\rightarrow e^{-i\beta}\bar\theta
\end{equation}
There is a slightly less trivial $R$-symmetry, by combining it with the other $\U(1)$ such that
\begin{equation}
	\phi\rightarrow\phi,\qquad \lambda^{2}\rightarrow e^{-i\beta}\lambda^{2},\qquad \bar\lambda^{2}\rightarrow e^{i\beta}\bar\lambda^{2},\qquad \lambda^{1}\rightarrow e^{i\beta}\lambda^{1},\qquad \bar\lambda^{1}\rightarrow e^{-i\beta}\bar\lambda^{1}
\end{equation}
Consider the Yukawa term
\begin{equation}
	\Tr \epsilon_{ij}\epsilon^{\alpha\beta}\comm{\lambda_{\alpha}^{i}}{\lambda_{\beta}^{j}}\bar\phi
\end{equation}
It was already antisymmetric in the lorentz indices $\alpha$ but is also antisymmetric in the gauge indices because the coupling came from the gauge coupling of the adjoint representation. Therefore, since the coupling of the gluons involve commutators of the generators of the Lie algebra, by extending with susy we get more commutators. So the superpartner to the gauge field get the commutator.\\
So this object is also antisymmetric in the internal indices. So by no added cost, the lagrangian has a bigger global symmetry $\SU(2)$ which acts only on fermions
\begin{equation}
	\mqty(\lambda^{1}\\\lambda^{2})\rightarrow M\mqty(\lambda^{1}\\\lambda^{2})
\end{equation}
with $\det M=1$. But we can relax this condition by taking
\begin{equation}
	\phi\rightarrow \det M\, \phi
\end{equation}
where now $M\in \U(2)$. So we actually have a $\U(2)$ global symmetry. Of course $\U(1)\times \U(1)$ is the maximal torus of $\U(2)$, the diagonal part of the subgroup which was manifest since the start.\\
The realization by $\cN=1$ enabled us to see the diagonal part of $\U(2)$. This $\U(2)$ symmetry does not commute with the $\cN=1$ susy. Afterall
\begin{equation}
	\delta A_{\mu}=\bar\epsilon \gamma_{\mu}\lambda^{1}+\cdot
\end{equation}
but we have a $\U(2)$ that mixes $\lambda^{1}$ with $\lambda^{2}$. So we have an added symmetry which puts the two on the same ground and we get the $\cN=2$ generalization
\begin{equation}
	\delta A_{\mu}=\sum_{i=1,2}\bar\epsilon_{i}\gamma_{\mu}\lambda^{i}+\cdots
\end{equation}
Of course one has to add all the other transformations.
\subsection{Dynamics}
We want to find the potential energy for the theory which is done by integrating the auxilliary field $D$ over the EOMs. By doing so one gets
\begin{equation}
	V(\phi,\bar\phi)=e^{2}\Tr\qty( i\comm{\phi}{\bar\phi}^{2})
\end{equation}
the $i$ makes the commutator hermitian so that the trace is positive. So
\begin{equation}
	V=0\iff \comm{\phi}{\bar\phi}=0
\end{equation}
so that the $\phi$ matrix commutes with it's hermitian which is equivalent to saying that the hermitian and antihermitian part of $\phi$ commute with each other. They can be simultaneously diagonalized by a gauge transformation. Consider $G=\SU(2)$ gauge for example
\begin{equation}
	\langle\phi\rangle=\mqty(a&0\\0&-a),\qquad a\in\bbC
\end{equation}
but $a$ is not quite gauge invariant because we can make another gauge transformation which exchanges the two eigenvalues. What is gauge invariant is 
\begin{equation}
	\Tr \phi^{2}=2a^{2}\equiv u
\end{equation}
Now $u$ is a compelx parameter that labels the possible classical vacua of the theory. But after quantum corrections it is also true that $u$ labelles the quantum vacuas.

In the complex $u$ plane, we have a special point of unbroken gauge symmetry $u=0$. But on a generic point the gauge group is broken to $\U(1)$. We understand well the $\U(1)$ gauge theories, which just generate Coulomb forces. In the other cases, we have strong gauge dynamics which is very difficult to understand. Because of asymptotic freedom, for large $u$ the classical picture is valid, but for small $u$ we have to understand the strong dynamics of the theory.
\subsection{Spectrum}
Consider $u\neq 0$. The massless states are the same theory for $\U(1)$: massless vector multiplet which has $8 $ states
\begin{table}[H]
\begin{tabular}{c|ccccc}
	\multirow{ 2}{*}{N. States} & - & - & 1 & 2 & 1  \\
	& 1 & 2 & 1 & - & - \\
	\hline
	\text{Helicity}&-1&$-\frac{1}{2}$&0&$\frac{1}{2}$&1
\end{tabular}
\end{table}
and then we have a massive $W^{+}$ boson which corresponds to strictly upper triangular matrices. On the other hand, the reps didn't become bigger just because they are massive; there are still $8$ states. So there has to be a central charge! [Olive, Witten; Phys Lett...]\\
One can calculate the anticommutator
\begin{equation}
	\acomm{Q_{\alpha}^{i}}{Q_{\beta}^{j}}=\frac{1}{e^{2}}\int\dd[3]x \partial_{i}(\phi F_{0i}^{+})
\end{equation}
where 
\begin{equation}
	F_{0i}^{+}=\frac{1}{2}\qty(F_{0i}+\frac{\epsilon_{ijk}}{2}F_{jk})\implies \vec{F_{0i}^{+}}=\frac{1}{2}\qty(\vec{E}_{i}+i\vec{B}_{i})
\end{equation}
Of course being an integral of a total derivative, one can write it as a surface term at infinity, but what are those surface terms? The electric and magnetic fields at infinity
\begin{equation}
	F_{0i}^{+}=\frac{1}{2}\qty(F_{0i}+\frac{\epsilon_{ijk}}{2}F_{jk})=a\qty(Q_{el}+\frac{i}{e^{2}}Q_{mag})	
\end{equation}
it is clear from the normalization is that the $e^{2}$ factor cancels for the electric chage, while it does not for the magnetic charge. Classialy
\begin{equation}
	Z=a\qty(Q_{el}+\frac{i}{e^{2}}Q_{mag})	
\end{equation}
therefore
\begin{equation}
	M\ge \abs{Z}=\abs{a}\sqrt{Q_{el}^{2}+\qty(\frac{Q_{mag}}{e^{2}})^{2}}
\end{equation}
For BPS (short) representations 
\begin{equation}
	M= \abs{Z}=\abs{a}\sqrt{Q_{el}^{2}+\qty(\frac{Q_{mag}}{e^{2}})^{2}}
\end{equation}
so for $W$-bosons
\begin{equation}
	M= \abs{Z}=\abs{a}\abs{Q_{el}}
\end{equation}
and this is just the standard Higgs formula, the mass of the $W$-boson is proportional to the value of the Higgs vev $a$.

The factor of $i$ in the central charge for the magnetic charge is very important since in lorentz signature the curvature field is complex and is relevant since in the mass, without the factor of $i$, we would have had a mixed term between electric and magnetic charge which is odd under CP since the two charges transform in an opposite way. We could have violated CP by turning on a $\theta$-angle. This would not spoil susy, but it the cross term would be there since CP is explicitly broken.

\section{\texorpdfstring{The other way of constructing $\cN=2$ theory}{Other Way}}
We are going to consider minimal SYM in $d$ dimensions. The only fields are the gauge field $A_{\mu}$ and the fermion $\lambda$. We take the smallest fermion in $d$ dimensions which is allowed by Lorentz invariance. The lagrangian is
\begin{equation}
	\cL = \frac{1}{e^{2}}\int \dd[4]{x}\Tr\qty(F_{\mu\nu}^{2}+\bar\lambda i\slashed{D}\lambda)
\end{equation}
If the lagrangian is supersymmetric it has to be invariant under
\begin{equation}
	\delta A_{\mu}=\epsilon\Gamma_{\mu}\lambda,\qquad \delta\lambda=\Gamma^{\mu\nu}F_{\mu\nu}\epsilon
\end{equation}
where $\Gamma_{\mu}$ is the Dirac matrices in $d$-dimensions. By plugging this variation, one can show that the term linear in $\lambda$ cancels in any dimensions
\begin{equation}
	4\int\dd[4]{x}\Tr \qty(F_{\mu\nu}D^{\mu}(i\bar\epsilon\Gamma^{\nu}\lambda)+\frac{1}{2}\bar\lambda i\Gamma^{\mu}D_{\mu}(\Gamma^{\alpha\beta}F_{\alpha\beta}\epsilon))=0
\end{equation}
This can be shown by using the algebra of the $\Gamma$ matrices, practically since
\begin{equation}
	\Gamma^{\mu}\Gamma^{\alpha\beta}=\Gamma^{\mu\alpha\beta}+g^{\mu\alpha}\Gamma^{\beta}-g^{\mu\beta}\Gamma^{\alpha}
\end{equation}
it decomposes in symmetric and anti-symmetric parts. Using Bianchi identity and integration by parts we can show that the two contributions cancel.

There is actually a part which is cubic in $\lambda$ because we can take the part in the gauge field inside $\slashed{D}$ and vary that
\begin{equation}
	\frac{\delta}{\delta A}\bar\lambda\Gamma^{\mu}\comm{A_{\mu}}{\lambda}=f^{abc}\lambda^{\alpha}_{a}\Gamma_{\mu\alpha\beta}\lambda^{\beta}_{c}(\Gamma^{\mu})_{\rho\sigma}\lambda_{c}^{\rho}\epsilon^{\sigma}
\end{equation}
The only hope for this to be zero is that
\begin{equation}
	\Gamma_{\mu\alpha\beta}\Gamma^{\mu}_{\rho\sigma}+\text{symm }\alpha\beta\sigma=0
\end{equation}
which is true only if $d=3,4,6,10$.\\
The $4d$ case is just $\cN=1$ SYM in four dimensions, for today we consider the $6d$ case. Therefore we have a $6d$ supersymmetric lagrangian but we wanted an $\cN=2$ in four dimensions. We just brutally take the fields independent of the last two dimensions
\begin{equation}
	t\,x^{1}\,x^{2}\,x^{3}\quad x^{4}\,x^{5}
\end{equation}
we let
\begin{equation}
	\phi=A_{4}+iA_{5},\qquad A_{\mu=0,1,2,3},\qquad \lambda=2\text{ spinors in }d=4 
\end{equation}
Note, $\cN=4$ in $4d$ can be constructed in the same way starting from $10d$.

This second construction makes it clear why there should be a central extension to the susy algebra
\begin{equation}
	\acomm{Q_{\alpha}}{Q_{\beta}}=\sum_{\mu=0}^{5}\sigma_{\alpha\beta}^{\mu}P_{\mu}=\sum_{\mu=0}^{3}\sigma_{\alpha\beta}^{\mu}P_{\mu}+\sum_{\mu=4}^{5}\sigma_{\alpha\beta}^{\mu}P_{\mu}
\end{equation}
but what are the two more terms? The momentum commutes with the momentum and the supercharges, and they commute with the $4d$ Lorentz transformations which is all that we got in $4d$. So they become central charges! In fact they become the electric charge.

Let us be more specific on the nature of $\lambda$. In $6d$ we would have $6$ gamma matrices that we can combine into three creation and annihilation operators
\begin{equation}
\begin{array}{ccc}
	\Gamma_{0}+i\Gamma_{1}&\Gamma_{2}+i\Gamma_{3}&\Gamma_{4}+i\Gamma_{5}\\
	\Gamma_{0}-i\Gamma_{1}&\Gamma_{2}-i\Gamma_{3}&\Gamma_{4}-i\Gamma_{5}	
\end{array}
\end{equation}
If one just tries to represent the Clifford algebra, there would be $8$ states. But we wanted the smallest fermion. In $6d$ we can ask that the chirality operator, product of all gammas, on $\Gamma^{0}\cdots\Gamma^{5}\lambda=+\lambda$ since 
\begin{equation}
	(\Gamma^{0}\cdots\Gamma^{5})^{2}=1
\end{equation}
but in four dimensions this squares to $-1$ so it's eigenvalues are $\pm i$ and are exchanged by CPT. In $4d$ the hermitian adjoint of the spinor is the other spinor. But in $6d$ the square is $+1$ so $\lambda$ can obey a chirality condition so half number of states $=4$. And one might think that $\lambda$ has only four components. It is true that $\SO(1,4)$ has a $4d$ representation, call it Spinor$_{+}$ bus it's pseudo-real. Since it is pseudo-real one needs to double it taking an $8d$ rep.\\
So the smallest rep has $8$ real components. So the susy generator $\epsilon$ also has $8=2\times 4$ components and so $\cN=2$. In four dimensions the relations is 
\begin{equation}
	(\Gamma^{0}\Gamma^{1}\cdots\Gamma^{3})\lambda=-\Gamma^{4}\Gamma^{5}\lambda
\end{equation}
and the same for $\epsilon$.

Let us consider $\lambda^{\alpha}$ as $\lambda^{a\,x}$ where $a=1,\ldots,4$ is an $\SO(1,5)$ index and $x$ there since we double it to make it real, so now we have $\SU(2)$ index.\\
So we could write the $\lambda$ part of the lagrangian in more details
\begin{equation}
	\epsilon_{xy}\lambda^{ax}\Gamma_{ab}^{\mu}D_{\mu}\lambda^{by}
\end{equation}
which is manifestly $\SO(1,5)$ and $\SU(2)$ invariant. We discovered an $\SU(2)$ symmetry which only acts on fermions! And we also have a $\U(1)_{R}$ symmetry which acts on bosons and fermions: set the charge of $A_{4},A_{5}$ to have charge $\pm1$ so that the fermion $\lambda$ has charge $\pm 1/2$. The $\SU(2)_{R}$ acts only on $\lambda$. To show that the model is actually supersymmetric we need the identity
\begin{equation}
	\lambda^{\alpha}\lambda^{\beta}\lambda^{\gamma}\gamma^{\mu}_{\alpha\beta}\gamma_{\mu\gamma\delta}\epsilon^{\delta}+(\alpha\beta\gamma)=0
\end{equation}
which with the explicit indices is
\begin{equation}
	\lambda^{a_{1}x_{1}}\lambda^{a_{2}x_{2}}\lambda^{a_{3}x_{3}}\ldots\epsilon^{a_{4}x_{4}}+(\alpha\beta\gamma)=0
\end{equation}
We have $4$ guys which transform in the $4d$ rep of $\SO(1,5)$, and the only invariant that we can make is with the completely anti-symmetric symbol $\epsilon_{a_{1}a_{2}a_{3}a_{4}}$.\footnote{It can be proven by using the fact that the complexified $\SO(1,5)$ is just $\SL(4)$} We are trying to make a completely symmetric object but there is a completely anti-symmetric object. So we have to anti-simmetrize in the $\SU(2)$ indices . But there is no way of doing it, so the object is zero and the model is supersymmetric.

We need to find the other term in the central charge of the susy algebra. There is another kind of small rep we need to understand for the model. Discuss things we can see at weak coupling. What can we understand of the spectrum of the model at weak coupling? We have two types of particles at weak coupling: thew ones coming from the oscillations (each field gives a particle), and then we can quantize classical lumps (classical solitons). We take our non-linear field equations and find a solution $\Phi(\vec{x})$ which is independent of time (We want classical ground states which are independent of time so tu avoid kinetic energy and make them stable) and we quantize.\\
A classical solution is some kind of stable lump. For simplicity we assume that the classical solution is unique except from what we can deduce by symmetry. We have a family of solution
\begin{equation}
	\Phi(\vec{x})=\Phi(\vec{x}+a)
\end{equation}
quantizing it means that we need to find a quantum wavefunction that depends on $a$. An interesting one is
\begin{equation}
	\Psi(a)=\exp\qty(i\vec{p}\cdot \vec{a})
\end{equation}
which describes a lump in a momentum eigenstate. It's energy can be found from the Hamiltonian
\begin{equation}
	H=\int\dd[4]{x}\qty(\frac{e^{2}}{2}\dot{\phi}^{2}+\frac{1}{2e^{2}}(\nabla\phi)^{2}+V(\phi))
\end{equation}
where the last two terms are just the mass. For stability, classically we choose solution which do not have kinetic energy but quantum mechanically the derivative becomes an operator
\begin{equation}
	-\frac{\hbar^{2}}{2}\qty(\pdv{}{\phi})^{2}
\end{equation}
and will act on all field variables. But the important ones are the one that do not contribute much to the energy: the zero modes $a$. This is makes the operator
\begin{equation}
	-\frac{\hbar^{2}}{2}C\qty(\pdv{}{a})^{2}
\end{equation}
so that
\begin{equation}
	H\Psi=\qty(M+\frac{p^{2}}{2M})\Psi
\end{equation}
The mass will be of order $1/e^{2}$. So near the classical limit, this object will be heavy and won't move a lot when is excited near it's classical state.

We repete the same argument with susy, so we need
\begin{equation}
	\delta\lambda=\Gamma^{\mu\nu}F_{\mu\nu}\epsilon=qty(\sum_{\mu,\nu=0}^{2}\Gamma^{\mu\nu}F_{\mu\nu}+\sum_{i=4,5}\Gamma^{\mu i}D_{\mu}A_{i}+\comm{A_{4}}{A_{5}}\Gamma^{\mu\nu})\epsilon
\end{equation}
Let us discuss when we find a totally random classical solution. In general $\delta\lambda$ will not be zero generally and we'll get $8$ fermion zero modes $\lambda_{1},\ldots,\lambda_{8}$. Then
\begin{equation}
	\lambda=\sum_{i=1}^{8}c_{i}\lambda_{i}+\text{ contributions from non-zero modes}
\end{equation}
We had our lagrangian
\begin{equation}
	I=\int \bar\lambda i\slashed{D}\lambda=\sum_{i=1}^{8}\int\dd{t}\qty(c_{i}\dv{c_{i}}{t}+0c^{2})+\text{ contributions from non-zero modes}
\end{equation}
The equivalent thing for quantizing of before, we suppose that $c_{i}$ can depend on time. When we quantize it we'll learn that $c_{i}$ is canonically conjugate of itself. This happens because we took a real basis. After quantization we get $2^{4}=16$ states. This is a generic supermultiplet. It's not interesting.

Let us pick coordinates in the internal space s.t $A_{5}\rightarrow 0$ at infinity but $A_{4}$ does not. We look for a special solution to the EOMs that has some special suspersymmetric properties (Bogomolny equations)
\begin{equation}
	F_{ij}=\frac{1}{2}\epsilon_{ijk}D_{k}A_{4}
\end{equation}
it's time independent so we just have space coordinates. These solutions carry magnetic charge (a magnetic monopole).For half of the $\epsilon$s then $\delta \lambda=0$. First of all
\begin{equation}
	F_{12}=D_{3}A_{4}
\end{equation}
So
\begin{equation}
	\delta\lambda\sim \qty(\Gamma^{12}F_{12}+\Gamma^{34}D_{3}A_{4})\epsilon=F_{}{12}(\Gamma^{12}+\Gamma^{34})\epsilon
\end{equation}•
which is zero if 
\begin{equation}
	\qty(\Gamma^{12}+\Gamma^{23})\epsilon=0\implies \Gamma^{1234}\epsilon=\epsilon
\end{equation}
This has many solutions. Half of the spinors obey this equation and the other half 
\begin{equation}
	\Gamma^{1234}\tilde\epsilon=-\tilde\epsilon
\end{equation}
Other terms lead to the same equation. So half the $\epsilon$ obey $\delta\lambda=0$ that means that there are $4$ fermionic zero modes $c_{i}$ and the action becomes
\begin{equation}
	\sum_{i=1}^{4}\int\dd{t}c_{i}\dot c_{i}
\end{equation}
which after quantizing gives the same algebra but with four operators. So the clifford algebra has dimension $2^{4/2}=4$ states. And this is a small representation. The CPT conjugate will double the spectrum with opposite magnetic charge.

So we found that a magnetic monopole has mass and momentum and has an important structure of an hypermultiplet. Moreover, a monopole can carry an electric charge. The classical solution is not invariant under (for $G=\SU(2)\rightarrow \U(1)$) the unbroken $\U(1)$. These rotations produce a new collective coordinate $\beta\in S^{1}$. So the classical solution 
\begin{equation}
	\Phi(x;a,\beta)
\end{equation}
where $a\in\bbR^{3}$ is the center of mass again and $\beta\in S^{1}$. The quantum wave function could be
\begin{equation}
	\Psi(a,\beta)=e^{ipa}e^{in\beta}
\end{equation}
but the eigenvalue of the rotation on the circle is the electric charge $n$. So we get monopoles with arbitrary integer electric charge.

There is a very important subtlety when we consider quantum mechanics on a circle. We assumed implicitly that $\Psi(\beta+2\pi)=\Psi(\beta)$. But we could as well assume  $\Psi(\beta+2\pi)=\Psi(\beta)e^{i\theta}$. A typical wavefunction would be
\begin{equation}
	\Psi(\beta)=\exp[i\qty(n+\frac{\theta}{2\pi})\beta]
\end{equation}
which is the same as we would have had with a theta angle.

We saw that for the $W$-boson was in a small representation for which
\begin{equation}
	Z=a Q_{el}
\end{equation}
but we saw that we also have magnetic monopoles in short representations with mass $M=\frac{4\pi^{2}}{e^{2}}\abs{a}$ and therefore this has to be in the susy algebra, such that
\begin{equation}
	Z=aQ_{el}+i\frac{4\pi}{e^{2}}a Q_{m}=a\qty(n_{e}+\frac{\theta}{2\pi}+n_{m})+\frac{4\pi i}{e^{2}}a n_{m}=n_{e} a +n_{m}\qty(\frac{\theta}{2\pi}+\frac{4\pi i}{e^{2}})a=n_{a}+n_{m}a\tau
\end{equation}
This is the classical answer, $\tau$ is subject to renormalization. We want to find the quantum equivalent of this formula.

Notice that $\cN=2$ SYM is asymptotically free, so is going to be simple for large $u$ but complicated for small $u$. For this theory the $\beta$-function is non zero, and therefore we have a chiral anomaly. In fact the $\U(1)_{R}$ is a classical symmetry but has a quantum anomaly from instantons. We just find the number of $\lambda$ zero-modes, which we know to be $8$ so that
\begin{equation}
	\U(1)\rightarrow \bbZ_{8}
\end{equation}
under which $u$ has charge $4$ so
\begin{equation}
	\bbZ_{8}:u\leftrightarrow -u
\end{equation}
By using this symmetry we can rotate the $\theta$ angle such that it is zero on the $\Re u$ axis $\theta_{eff}=2\arg{u}$. Now one can consider the electric charge of a magnetic monopole, our formula tells us that if we increase $\theta$ by $2\pi$, the electric charge increases by $1$. The electric charge are all possible integers, when we move half way into the $u$ plane, it will be shifted by one and when we do a full rotation it will be shifted by $2$. Therefore there are two kinds of monopoles: even and odd electric charge and under monodromy around infinity those two groups get exchanged with one another.

\section{The third construction}
To understand the theory quantum mechanically we use the following construction. We work in superspace with two supersymmetries $\theta_{\alpha}^{i},\tilde\theta_{\dot\beta j}$. In this space, susy is realized as
\begin{equation}
	Q_{\alpha i}=\pdv{\theta^{\alpha i}}+i\tilde\theta^{\dot\alpha}_{i}\gamma^{\mu}_{\alpha\dot\alpha}\pdv{}{x^{\mu}}
\end{equation}
and similarly for $\tilde Q$. The key to being able to writing supersymmetric theories is to find the operators which commute with susy
\begin{equation}
	D_{\alpha i}=\pdv{\theta^{\alpha i}}.i\tilde\theta^{\dot\alpha}_{i}\gamma^{\mu}_{\alpha\dot\alpha}\pdv{}{x^{\mu}}
\end{equation}
and similarly with $\tilde D_{\dot\alpha}^{j}$. So susy theoreis are constructed by using the $D$s (also $\partial_{\mu}$). Then we want to do gauge theories in superspace. Naively speaking we just add superfields, by replecing the derivatives with covariant derivatives
\begin{equation}
	\cD_{\alpha i}=D_{\alpha i}+A_{\alpha i}(x,\theta,\tilde\theta),\qquad \tilde\cD_{\dot\alpha}^{i}=\tilde D_{\dot\alpha}^{i}+A_{\dot\alpha}^{i}(x,\theta,\tilde\theta)
\end{equation}
and similarly for $D_{\mu}$. But this is not quite what we want: it has too many fields
\begin{equation}
	\acomm{\cD_{\alpha i}}{\tilde\cD_{\dot\alpha}^{j}}-\delta^{i}_{j}\sigma^{\mu}_{\alpha \dot\alpha}D_{\mu}=\cP_{\alpha\dot\alpha,j}^{i}=0
\end{equation}
and this is a dimension one lorentz vector which is gauge covariant and has $R$-symmetries. But there is nothing like this in $\cN=2$ sYM. So we set it to zero. We get a supersymmetric condition. Suppose now 
\begin{equation}
	\acomm{\tilde\cD_{\dot\alpha}^{i}}{\tilde\cD_{\dot\beta}^{j}}=\Phi^{ij}_{\dot\alpha\dot\beta}=\epsilon_{\dot\alpha\dot\beta}\epsilon^{ij}\Phi
\end{equation}
we don't want the symmetric part and we get a scalar field in the adjoint, which is what we want; $\phi=A_{4}+iA_{5}$ will be the lowest component of $\Phi$.\\
One more beautiful fact is
\begin{equation}
	\tilde\cD_{\dot\alpha}^{i}\Phi=0
\end{equation}
so is chiral. Hence we can pick an holomorphic function $\cF(\Phi)$ and write an action
\begin{equation}
	I=\Im \int\dd[4]{x}\dd{\theta_{\alpha}^{i}}\cF(\Phi)
\end{equation}

Consider the microscopic definition of the $\SU(2)$ theory where
\begin{equation}
	\cF(\Phi)=\tau\Tr \Phi^{2}
\end{equation}
After the doing the integral in superspace, one gets the same lagrangian as before. What we need now, is the infrared theory: far from the origin in $u$ space we have a classical picture where we have a single $\Phi=a+\theta\lambda+\theta\sigma^{\mu\nu}\theta F_{\mu\nu}+\cdots+\theta^{4}\Box \bar a$ field. Since the low ebergy gauge group is $\U(1)$ and the adj rep is trivial, any holomorphic function $\cF(\Phi)$ makes sense and the low energy theory is given by 
\begin{equation}
	I=\Im\int\dd[4]{x}\dd[2]\theta \cF(\Phi)
\end{equation}
If we find $\cF(\Phi)$ we understood the theory.
\section{The Seiberg--Witten construction}
Classically 
\begin{equation}
	\cF(\Phi)=a^{2}\tau_{cl}
\end{equation}
But there is a 1-loop correction to $\tau$
\begin{equation}
	\frac{i}{2\pi}a^{2}\log \frac{a^{2}}{\Lambda_{0}^{2}}
\end{equation}
This one loop correction contains asymptotic freedom. By shifting $\Lambda_{0}$ we can reabsorb the $\tau_{cl}$ term so that
\begin{equation}
	\tau=\frac{i}{2\pi}a^{2}\log\frac{a^{2}}{\Lambda^{2}}
\end{equation}

Whatever $\cF$ is, given the expansion of the field $\Phi$, let us find some interesting terms
\begin{equation}
	I=\int\dd[4]{x}\pdv{\cF(\Phi)}{a}\Box \bar{a}+\pdv[2]{\cF(\Phi)}{a}(F^{+}_{\mu\nu})^{2}+\text{fermions }
\end{equation}
So in general, $\tau(a) =\pdv[2]{\cF}{a}$. But we need $\tau$ to have a positive immaginary part which is impossible since the imaginary part of holomorphic functions cannot be positive everywhere (minimal modulus principle for holomorphic functions).

Let us see some thing that work. In the $1$-loop approximation we find
\begin{equation}
	\pdv[2]{\cF}{a}=\frac{i}{\pi}\log\frac{a^{2}}{\Lambda^{2}}+\ldots=\frac{\theta_{eff}}{2\pi}+\frac{4\pi i}{e^{2}_{eff}}
\end{equation}
so that
\begin{equation}
	\frac{4\pi}{e_{eff}^{2}}=\frac{1}{\pi}\log\abs{\frac{a^{2}}{\Lambda^{2}}}
\end{equation}
so we have asymptotic freedom. Moreover
\begin{equation}
	\frac{\theta_{eff}}{2\pi}=-\frac{2}{\pi}\Im\log a
\end{equation}
On the $u$ plane $u=a^{2}$, so if $u$ goes with a monodromy at infinity, then $a$ goes with half a monodromy. Therefore $\log a$ picks up a factor of $i\pi$. So
\begin{equation}
	\frac{\theta_{eff}}{2\pi}\rightarrow \frac{\theta_{eff}}{2\pi}+2
\end{equation}
We know also that there are no perturbative corrections to the $\tau$. But the formula is no good. So we also need to account for non-perturbative corrections from instantons which carry factors of $\theta$. Even with non-perturbative corrections we would be in trouble for the same statement as before.

We used the fact that the low energy description is not unique: $\tau\rightarrow\tau +1$ or $\theta\rightarrow \theta+2\pi$. But there is another ambiguity: electro-magnetic duality. Classically $E\leftrightarrow B$, but quantumechanically it acts as 
\begin{equation}
	\tau\rightarrow\frac{1}{\tau}
\end{equation}
The part of the action that gave the kinetic energy for the scalars is
\begin{equation}
	\Im\int\dd[4]{x}\qty(\pdv{\cF}{a}\Box \bar a)=\Im\int\dd[4]{x}\pdv[2]{\cF}{a}\partial_{\mu}a\partial^{\mu}\bar a
\end{equation}
so that the Kähler metric in $a$ space is just
\begin{equation}
	\Im \pdv{\cF}{a}\dd{a}\dd{\bar a}
\end{equation}
and we are in trouble if the imaginary part of the derivative is not positive. We introduced the dual function 
\begin{equation}
	a_{D}(a)=\pdv{\cF}{a}\implies \tau=\pdv{a_{D}}{a}
\end{equation}
But notice that the action can be rewritten as
\begin{equation}
	\int\dd[4]{x}\partial_{\mu}{a_{D}}\partial^{\mu}{\bar a}
\end{equation}
So we got rid of $\cF$ at the cost of adding the dual variable. Under EM duality
\begin{equation}
	\mqty(a_{D}\\a)\rightarrow\mqty(a\\-a_{D})
\end{equation}
so that
\begin{equation}
	\dv{a_{D}}{a}\rightarrow -\frac{1}{\dv{a_{D}}{d}}
\end{equation}
which is what we expect on the coupling.

So we have two equivalent description 
\begin{equation}
	(a,a_{D},\tau)\leftrightarrow (a_{D},-a,-\frac{1}{\tau})
\end{equation}
and the two theories have to different photons.

With this description we can rewrite the central charge as
\begin{equation}
	Z=n_{e}a+n_{m}a_{D}
\end{equation}
This is renormalization group invariant since the two fields are meaningful in the low energy theory. It is also EM invariant.\\
Consider also the action of the shift on $\tau$ on these fields
\begin{equation}
	\mqty(a_{D}\\a)\rightarrow \mqty(a_{D}+a\\a)=\mqty(1&1\\0&1)\mqty(a_{D}\\a)
\end{equation}
Together with EM duality
\begin{equation}
	\mqty(a_{D}\\a)\rightarrow \mqty(a\\-a_{D})=\mqty(0&1\\-1&0)\mqty(a_{D}\\a)
\end{equation}
we have an action of $\SL(2,\bbZ)$. The low energy effective description of this theory by a single vector multiplet of $cN=2$ susy is only unique up to an $\SL(2,\bbZ)$ transformation.

The central charge has to be independent of which description so
\begin{equation}
	\mqty(a_{D}\\a)\rightarrow M\mqty(a_{D}\\a)\implies \mqty(n_{m}&n_{e})\rightarrow \mqty(n_{m}&n_{e})M^{-1}
\end{equation}
If $M\in\SL(2,\bbZ)$ also $M^{-1}$ has integer values. So the transformation acts by integers on the electric and magnetic charges.

The classical answer in the $u$ plane was that there was one bad point. The monodromy at infinity is $u\rightarrow e^{2\pi i}u$ which implies
\begin{equation}
	\mqty(a_{D}\\a)\rightarrow \mqty(-a_{D}+2a\\-a)
\end{equation}
where the additional factor is due to the fact that the $\theta$ angle changes around the monodromy ad infinity. So the monodromy at infinity is
\begin{equation}
	\mqty(a_{D}\\a)\rightarrow \mqty(-1&2\\0&-1)\mqty(a_{D}\\a)
\end{equation}
with the property that $a^{2}$ is invariant. If there was only one bad point, then the monodromy at infinity would be the only monodromy and $a^{2}$ would be the right answer. But we know that this answer is not correct do to the non positivity of the imaginary part of $\tau$ as a prepotential. So there must be other singularities so that there can be non-abelian monodromies: there will be at least two. Since there is a symmetry $u\rightarrow-u$ then neither one will be at zero. So the minimal picture has two singularities which are at non-zero values of $u$, call them $\pm\Lambda^{2}$. From the classical point of view, $\Lambda$ is exponentially small. By assuming that there are only two bed points, one gets the right answer (there is also confinement as a reasoning).

What would be a bad point in physics: first order phase transition. But here is much more reasonable to think that massless particles appear. So the nature of the singularity is that there are extra massless particles. We assume that they are hypermultiplets but they should not have only electric charges. Only electric massless hypermultiplets does not help. The only hypermultiplets that we can find in the theory are magnetically charged hypermultiplets that could have also an electric charge. So the hypotesis is the the magnetic monopoles become massless at these points.\\
Take a monopole $n_{m}=1,n_{e}=0$ with mass $M$ that goes to zero mass. Probably the best description is in the magnetic dual. Here the monopoles are electrically charged. Here QED is not IR free. So there is a $1$-loop correction
\begin{equation}
	\tau_{D}=\dv{a}{a_{D}}=-\frac{i}{\pi}\log a_{D},\quad \text{where }a_{D}\rightarrow 0
\end{equation}
Around $a_{D}=0$, $\tau_{D}$ has a monodromy $\tau_{D}\rightarrow \tau_{D}-2$ and $a\rightarrow a-2a_{D}$ so
\begin{equation}
	\mqty(a_{D}\\a)\rightarrow \mqty(1&0\\-2&1)\mqty(a_{D}\\a)
\end{equation}
So
\begin{equation}
	M_{\infty}= \mqty(-1&2\\0&-1),\quad M_{\Lambda^{2}}=\mqty(1&0\\-2&1),\quad M_{\infty}=M_{\Lambda^{2}}M_{-\Lambda^{2}}\implies M_{-\Lambda^{2}}= \mqty(1&-2\\-2&3)
\end{equation}
The last one is the monodromy due to a massless dyon.

We found a flat $\SL(2,\bbZ)$ bundle on the $u$ plane with two points removed.
