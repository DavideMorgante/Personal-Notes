\chapter{Chern-Simons theory}
This chapter is basically a relaboration and expansion of \cite[Chapter 2]{Moor} which has also been covered in a four lectures series at TASI by Greg Moore on CS-theory. 

\section{\texorpdfstring{$U(1)$-Chern Simons Theory}{CS}}
A Chern-Simons theory is a Gauge theory with the following inputs
\begin{enumerate}
    \item A \textbf{Compact Lie group $G$}\footnote{Generalizations to Finite groups will be discussed In chapter \ref{ZNCS}}
    \item An \textbf{orientable $3$-Manifold} $\mathcal{M}_3$
    \item A \textbf{$G$-Principal Bundle $P \to \mathcal{M}$} with local connection $A_i \in \Omega(U_i) \otimes \mathfrak{g}$ ($\cup_{i} U_i = \mathcal{M}_3$)
\end{enumerate}
If the bundle $P \to \mathcal{M}_3$ can be trivialized\footnote{$P \simeq G \times \mathcal{M}_3$}, $\{A_i\}$ can be consistently patched together in a well defined global $\mathfrak{g}$-valued $1$-form $A$. The Chern-Simons action is:
\begin{equation}
    S_{\text{Chern-Simons}} = \frac{k}{8 \pi^2} \int_{\mathcal{M}_3} \text{Tr}(A \wedge d A + \frac{2}{3} A \wedge A \wedge A)
\end{equation}

\section{Quantization}
\subsection{Quantization Procedures}

\subsection{Mathematical Aside: Moment Map and Symplectic reduction}
Let $(\mathcal{M}, \omega)$ be a \textbf{symplectic manifold}. We are interested in studying the action $\phi_{(\cdot)}$ of some Lie group $G$ on $\mathcal{M}$ which preserves the symplectic form $\omega$:
\begin{equation}
	\label{sy}
    \phi_g^*(\omega) = \omega
\end{equation}
for any $g \in G$. Flows of this type are called \textbf{symplectomorphisms}. A simple example of those actions is provided by \textbf{Hamiltonian Vector Field} $X_f \in \Gamma(T \mathcal{M})$ associated to any $f \in C^{\infty}(\mathcal{M})$. Those vector fields are defined by the condition:
\begin{equation}
    \omega(X_f,\cdot) = \dd f
\end{equation}
which can be read in coordinates $(x^1,\cdots,x^{2n})$, as
\begin{equation}
    \omega_{ij} (X_f)^i = \partial_j f \to X_f^i = \omega^{ij} \partial_j f 
\end{equation}
where $\omega^{ij} \in \Gamma(T \mathcal{M} \otimes T \mathcal{M} )$ such that $\sum_j \omega^{ij} \omega_{jk} = \delta^{i}_k$. If we consider the group of diffeomorphisms generated by $X_f$, $\Phi_{X_f}^t$ ($t \in \mathbb{R}$), the condition \eqref{sy} is indeed satisfied because at any point $p \in \mathcal{M}$
\begin{equation}
   \mathcal{L}_{X_f} (\omega) = d \omega(X_f, \cdot) +  \cancel{(d \omega)}(X_f, \cdot , \cdot) = d^2 f =0
\end{equation}

In general, if $\xi \in \Gamma(T \mathcal{M})$ is the generator of the one group-parameter of symplectomorphism $\phi_{\exp(t T)}$ along some direction given by $T \in \mathfrak{g}$, using the same reasoning as above, it holds that:
\begin{equation}
    d \omega(\xi, \cdot) =0
\end{equation}
If $H^1_{\text{dR}}(\mathcal{M})= \{0\}$, then we can find a function $\mu(T) \in C^{\infty}(\mathcal{M})$ such that
\begin{equation*}
    \dd \mu(T) = - \omega(\xi, \cdot)
\end{equation*}
$\mu(T)$ is called the \textbf{charge} associated to the symmetry generated by the vector field $\xi$, because it generates $\phi_{\exp(t T)}$. Indeed
\begin{equation}
    \{ \mu(T), f\} = \omega^{ij} \partial_i \mu(T)\partial_j f = - \omega^{ij} \omega_{ik} \xi^i \partial_j f = \delta^j_k \xi^i \partial_j f = \xi^j \partial_j f = \xi(f) = \mathcal{L}_{\xi}(f)
\end{equation}
In more formal terms
\begin{defn}{Momentum Map}{}
   A Momentum Map is a linear assignment
  \begin{equation}
  \begin{aligned}
  	 \mu:  \mathfrak{g} &\to C^{\infty}(\mathcal{M})\\
	T &\mapsto \mu(T)
    \end{aligned}
\end{equation}
such that
\begin{equation}
     \phi_g^* \mu = \text{Ad}^*_g \mu
\end{equation}
where $\text{Ad}^*_g$ is the \textbf{coadjoint action} of $G$ on $\mathfrak{g}^*$ defined by
\begin{equation}
    \left \langle \text{Ad}^*_g \mu , T \right \rangle = \left \langle \mu , \text{Ad}_g T \right \rangle
\end{equation}
Equivalently, $\mu \in  C^{\infty}(\mathcal{M}) \otimes \mathfrak{g}^*$ by usual identification
\begin{equation}
    \left \langle \mu, T \right \rangle = \mu(T)
\end{equation}
\end{defn} 
Let's see some examples of momentum map.

\section{\texorpdfstring{$\mathbb{Z}_n$-Chern Simons theory}{ZnCS}}\label{ZNCS}
